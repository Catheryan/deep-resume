%%%%%%%%%%%%%%%%%%%%%%%%%%%%%%%%%%%%%%%
% Deedy - One Page Two Column Resume
% LaTeX Template
% Version 1.2 (16/9/2014)
%
% Original author:
% Debarghya Das (http://debarghyadas.com)
%
% Original repository:
% https://github.com/deedydas/Deedy-Resume
%
% IMPORTANT: THIS TEMPLATE NEEDS TO BE COMPILED WITH XeLaTeX
%
% This template uses several fonts not included with Windows/Linux by
% default. If you get compilation errors saying a font is missing, find the line
% on which the font is used and either change it to a font included with your
% operating system or comment the line out to use the default font.
% 
%%%%%%%%%%%%%%%%%%%%%%%%%%%%%%%%%%%%%%
% 
% TODO:
% 1. Integrate biber/bibtex for article citation under publications.
% 2. Figure out a smoother way for the document to flow onto the next page.
% 3. Add styling information for a "Projects/Hacks" section.
% 4. Add location/address information
% 5. Merge OpenFont and MacFonts as a single sty with options.
% 
%%%%%%%%%%%%%%%%%%%%%%%%%%%%%%%%%%%%%%
%
% CHANGELOG:
% v1.1:
% 1. Fixed several compilation bugs with \renewcommand
% 2. Got Open-source fonts (Windows/Linux support)
% 3. Added Last Updated
% 4. Move Title styling into .sty
% 5. Commented .sty file.
%
%%%%%%%%%%%%%%%%%%%%%%%%%%%%%%%%%%%%%%%
%
% Known Issues:
% 1. Overflows onto second page if any column's contents are more than the
% vertical limit
% 2. Hacky space on the first bullet point on the second column.
%
%%%%%%%%%%%%%%%%%%%%%%%%%%%%%%%%%%%%%%


\documentclass[]{deedy-resume-openfont}
\usepackage{fancyhdr}
    
\pagestyle{fancy}
\fancyhf{}
    
\begin{document}

%%%%%%%%%%%%%%%%%%%%%%%%%%%%%%%%%%%%%%
%
%     LAST UPDATED DATE
%
%%%%%%%%%%%%%%%%%%%%%%%%%%%%%%%%%%%%%%
\lastupdated

%%%%%%%%%%%%%%%%%%%%%%%%%%%%%%%%%%%%%%
%
%     TITLE NAME
%
%%%%%%%%%%%%%%%%%%%%%%%%%%%%%%%%%%%%%%
\namesection{颜正浩}{}{ \urlstyle{same}\href{mailto:741699287@qq.com}{741699287@qq.com} | Android开发工程师 }

%%%%%%%%%%%%%%%%%%%%%%%%%%%%%%%%%%%%%%
%
%     COLUMN ONE
%
%%%%%%%%%%%%%%%%%%%%%%%%%%%%%%%%%%%%%%

\begin{minipage}[t]{0.25\textwidth} 

%%%%%%%%%%%%%%%%%%%%%%%%%%%%%%%%%%%%%%
%     EDUCATION
%%%%%%%%%%%%%%%%%%%%%%%%%%%%%%%%%%%%%%

\section{教育经历} 
\sectionsep

\subsection{南京理工大学紫金学院}
\descript{本科,计算机科学与技术}
\location{2014.09-2018.06}
\sectionsep



%%%%%%%%%%%%%%%%%%%%%%%%%%%%%%%%%%%%%%
%     LINKS
%%%%%%%%%%%%%%%%%%%%%%%%%%%%%%%%%%%%%%

\section{开源及分享}
\sectionsep
博客:  \href{https://catheryan.github.io/}{\bf Catheryan-Blog} \\   
Github:// \href{https://github.com/Catheryan/}{\bf catheryan} \\
yuque://  \href{https://www.yuque.com/g/juejiangdexiaohao/fflqe5/collaborator/join?token=98fARMSD1wPewTat&source=book_collaborator# 《项目整理》}{\bf 知识分享} \\

%%%%%%%%%%%%%%%%%%%%%%%%%%%%%%%%%%%%%%
%     COURSEWORK
%%%%%%%%%%%%%%%%%%%%%%%%%%%%%%%%%%%%%%

% \section{修读课程}
% \subsection{Graduate}
% Advanced Machine Learning \\
% Open Source Software Engineering \\
% Advanced Interactive Graphics \\
% Compilers + Practicum \\
% Cloud Computing \\
% Evolutionary Computation \\
% Defending Computer Networks \\
% Machine Learning \\
% \sectionsep

%%%%%%%%%%%%%%%%%%%%%%%%%%%%%%%%%%%%%%
%     SKILLS
%%%%%%%%%%%%%%%%%%%%%%%%%%%%%%%%%%%%%%

\sectionsep
\section{技能}
\sectionsep
\subsection{Android}
\sectionsep
\location{主导多个大型项目模块上线}
\sectionsep
\subsection{语言}
\sectionsep
Kotlin \textbullet{} \\
Java \textbullet{} \\ 
C++ \textbullet{} \\
Javascript \\
\sectionsep

\subsection{跨端技术}
\sectionsep
Kotlin Multplatform \textbullet{} \\
Rax \textbullet{Vue}\\
\sectionsep

\subsection{DevOps}
\sectionsep
gradle \textbullet{ci·cd}\\
Jenkins \textbullet{} 自动化测试

%%%%%%%%%%%%%%%%%%%%%%%%%%%%%%%%%%%%%%
%
%     COLUMN TWO
%
%%%%%%%%%%%%%%%%%%%%%%%%%%%%%%%%%%%%%%

\end{minipage} 
\hfill
\begin{minipage}[t]{0.73\textwidth} 

%%%%%%%%%%%%%%%%%%%%%%%%%%%%%%%%%%%%%%
%     EXPERIENCE
%%%%%%%%%%%%%%%%%%%%%%%%%%%%%%%%%%%%%%

\section{从业经历}
\sectionsep
\runsubsection{腾讯·OTT视频业务}
\descript{实习}
\vspace{0.2em}
\location{2017.06 - 2017.09 | 深圳}
\vspace{\topsep}
\begin{tightemize}
    \item 期间主要负责OTT业务迭代以及图片库调研 \textbf{并采纳}
    \item 合作一份实用性专利 \textbf{已归档}
\end{tightemize}
\sectionsep

\runsubsection{阿里巴巴·大文娱}
\descript{优酷·基础业务Android开发工程师}
\vspace{0.2em}
\location{2018.06 - 2021.01 | 北京}
\begin{tightemize}
    \item 主要负责基础业务以及大会员业务,基础功能组件维护
    \item 期间多次获得高潜评价
\end{tightemize}
\sectionsep

\runsubsection{哔哩哔哩·商业}
\descript{Android开发工程师}
\vspace{0.2em}
\location{2021.02 - 至今 | 上海}
\begin{tightemize}
    \item 商业化交易模块对接人,负责业务迭代以及技术攻关
    \item 前期负责工程设计以及业务攻关,交易流程sdk化、渠道化,ci化 \textbf{公司各渠道并运行}
    \item 后期广告模块业务,负责业务迭代以及技术攻关
\end{tightemize}
\sectionsep

%%%%%%%%%%%%%%%%%%%%%%%%%%%%%%%%%%%%%%
%     RESEARCH
%%%%%%%%%%%%%%%%%%%%%%%%%%%%%%%%%%%%%%

\section{项目经历-核心}
\sectionsep
\runsubsection{\href{}{\bf 跨端链式事务处理}}
\runsubsection{Owner}
\descript{基于KMM的链式点击处理框架}
\vspace{0.2em}
\location{2024.12 - 2025.02}
\begin{tightemize}
    \item 主要利用\textbf{Kotlin Multiplatform}技术栈,提供跨端的链式点击处理能力
    \item 提供链式点击日志能力 \textbf{线上稳定运行,便于业务排查}
    \item 节点组合能力,逻辑统一
    \end{tightemize}
\sectionsep

\runsubsection{\href{}{\bf 动态化落地}}
\runsubsection{Owner}
\descript{基于redwood的跨端动态化框架}
\vspace{0.2em}
\location{2025.03 - 2025.05}
\begin{tightemize}
    \item 从本地Native到跨端的动态化能力,实现简单卡片落地
    \item 利用Zipline桥接能力,串联KMM业务层能力
    \end{tightemize}
\sectionsep

\runsubsection{\href{}{\bf InAppMessage}}
\runsubsection{Owner}
\descript{基于轮训结构的站内信系统}
\vspace{0.2em}
\location{2021.11 - 2022.01}
\begin{tightemize}
    \item 基于页面维度触发站内信消息,提供跨进程\textbf{main和web}的消息传递能力
    \item 同时也在探索基于端智能能力探索,提供更好的消息触达能力
    \end{tightemize}
\sectionsep

\runsubsection{\href{}{\bf TradeSdk}}
\runsubsection{Owner}
\descript{基于gradle的交易流程工程化}
\vspace{0.2em}
\location{2023.07 - 2023.9}
\begin{tightemize}
    \item 主要利用gradle的插件化能力,提供交易业务的sdk能力,提供给各渠道
    \item 提供config方案,平稳从web过度到Native渠道能力
    \end{tightemize}
\sectionsep

\runsubsection{\href{}{\bf HalfRoute}}
\runsubsection{Owner}
\descript{基于Fragment的Android半屏路由框架}
\vspace{0.2em}
\location{2023.04 - 2023.06}
\begin{tightemize}
    \item 主要利用Fragment的生命周期能力,提供半屏路由能力
    \item 对全屏路由能力的补充
    \end{tightemize}
\sectionsep


%%%%%%%%%%%%%%%%%%%%%%%%%%%%%%%%%%%%%%
%     OPEN SOURCE
%%%%%%%%%%%%%%%%%%%%%%%%%%%%%%%%%%%%%%

\section{开源贡献}
\begin{tabular}{ll}
\href{https://github.com/Catheryan/HoneyBee-Plugins}{\bf HoneyBee-Plugins} & 对路由注解核心机制的拆解 \\
\end{tabular}
\sectionsep

%%%%%%%%%%%%%%%%%%%%%%%%%%%%%%%%%%%%%%
%     SELF-ASSESSMENT
%%%%%%%%%%%%%%%%%%%%%%%%%%%%%%%%%%%%%%

\section{自我评价}
\sectionsep
\begin{tightemize}
    \item 热爱技术,热爱分享,热爱开源
    \item 具备良好的沟通能力和团队协作能力
\end{tightemize}

%%%%%%%%%%%%%%%%%%%%%%%%%%%%%%%%%%%%%%
%     PUBLICATIONS
%%%%%%%%%%%%%%%%%%%%%%%%%%%%%%%%%%%%%%

% \section{Publications} 
% \renewcommand\refname{\vskip -1.5cm} % Couldn't get this working from the .cls file
% \bibliographystyle{abbrv}
% \bibliography{publications}
% \nocite{*}

\end{minipage} 
\end{document}  \documentclass[]{article}
